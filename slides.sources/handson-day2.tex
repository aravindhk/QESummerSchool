\documentclass[landscape]{foils}
\usepackage[pdftex]{color}
\usepackage[pdftex]{graphicx}
\usepackage{listings}
\usepackage{amsmath}
\usepackage{hyperref}

\input{aliases}

\begin{document}
\blue

%%%%%%%%%%%%%%%%%%%%%%%%%%%%%%%%%%%%%%%%%%%%%%%%%%%%%%%%%%%%% 
\Head{QE-2019: Hands-on session -- Day-2}
Topics of Day-2 hands-on session:
\begin{itemize}
\item Structural optimizations:
  \begin{itemize}
  \item relaxations -- atomic positions only (\file{example1.relax})
  \item variable-cell relaxations (\file{example2.vc-relax})
  \end{itemize}
\item NEB method: saddle points of elementary chemical
  reactions (\file{example3.neb/})
\item Advanced functionals (\file{example4.functionals/})\\
  (slides for advanced-functionals are provided by \file{handson-day2-functionals.pdf})
\item Automating the workflow with PWTK
  (\file{example5.pwtk})
\end{itemize}

%%%%%%%%%%%%%%%%%%%%%%%%%%%%%%%%%%%%%%%%%%%%%%%%%%%%%%%%%%%%%
\head{1. How to perform structural optimization: graphane}
\rightheader{}
\rightfooter{Example: \file{Day-2/example1.relax}}

\parbox{17cm}{
  \begin{itemize}
  \item Move to \file{Day-2/example1.relax/} directory.\\[0.5em]
    Graphane is like graphene, with an H atom bound to each C atom in
    {\em trans} configuration. You need to optimize atomic positions,
    i.e., find the minimum-energy structure (zero forces).
  \end{itemize}
} \hskip 1cm
\parbox{8cm}{ \includegraphics[width=8cm]{figs/graphane.pdf}}

\begin{itemize}
\item File \file{pw.graphane.relax.in} is a modified version of\\
  \file{pw.graphene1x1.scf.in} with:
  \begin{itemize}
  \item \var{calculation='relax'} for structural optimization and a
    new namelist \nml{\&IONS} with variable \var{upscale=100.0}
  \item \var{ntyp=2} (2 types of atoms), \var{nat=4}
    (4 atoms in the cell)
  \item \card{ATOMIC\_SPECIES} card with 2 species of atoms and
    pseudopotentials
  \item \card{ATOMIC\_POSITION} card with 4 initial positions
    (C--H distance $\sim 1$~\AA)
  \end{itemize}
\end{itemize}

%%%%%%%%%%%%%%%%%%%%%%%%%%%%%%%%%%%%%%%%%%%%%%%%%%%%%%%%%%%%% 
\head{1. How to perform structural optimization: graphane (II)}
\begin{itemize}
\item Run the structural optimization, i.e.:\\[0.5em]
  \exec{pw.x < pw.graphane.relax.in > pw.graphane.relax.out \&}
\item When calculation finishes, analyze the output: it consists of
   several SCF steps, followed by calculation of forces and generation
   of new atomic positions.
\item To visualize the evolution of the structure during structural
   optimization, execute:\\[1em]
   \exec{xcrysden --pwo pw.graphane.relax.out}
\end{itemize}
\parbox{15cm}{Relaxed structure of graphane exhibits ``buckling''.}
\hskip 0.5cm \parbox{8cm}{
  \includegraphics[width=8cm]{figs/graphane2.pdf}
}

%%%%%%%%%%%%%%%%%%%%%%%%%%%%%%%%%%%%%%%%%%%%%%%%%%%%%%%%%%%%% 
\head{1. Supercell and structural optimization: graphene-oxide}

\parbox{15cm}{
  \begin{itemize}
  \item The first stage of graphene oxidation is the formation of an
    epoxy bridge. Let us add an O atom on a $(3\times3)$ supercell of
    graphene
  \end{itemize}
}\hskip 2cm\parbox{7cm}{
  \begin{flushright}
    \includegraphics[width=7cm]{figs/graphene3x3-O.pdf}    
  \end{flushright}
}

This example consists of two tasks: (i) build a $(3\times3)$ supercell
of graphene; (ii) add an O atom onto graphene--$(3\times3)$ supercell
structure and run a relaxation calculation.
    
{\bf Step-1:}
\vspace{-1em}
\begin{itemize}
\item The $(3\times3)$ supercell structure of
  graphene is provided in file\\
  \file{pw.graphene3x3.scf.in}, which is a modified version of\\
  \file{pw.graphene1x1.scf.in}. Please
  notice that:
  \begin{itemize}
  \item Lattice parameters $a$ and $b$ are multiplied by 3:
    \var{celldm(1)=13.962}
    \vspace{0.3em}
  \item Lattice parameter $c$ remains the same, hence \var{celldm(3)},
    which equals $c/a$, is divided by 3, hence: \var{celldm(3)=1.0}.
    \vspace{0.3em}
  \item There are 9 times the atoms of the original unit cell, i.e. \var{nat=18}.
    \vspace{0.3em}
  \item Reciprocal lattice vectors in the $xy$ plane are divided by 3
    (look at the output), hence if you want the same k-point grid,
    just use \card{K\_POINTS (automatic)} with \card{3 3 1 0 0 0}
    grid, which is equivalent to \card{9 9 1 0 0 0} k-point grid of
    $(1\times1)$ unit-cell \vspace{0.3em}
  \item Provided that {\red the two k-point grids are equivalent}:
    \begin{itemize}
    \item
      The energy of the supercell $E^{\rm SC} = 9E^{\rm UC}$ almost exactly
      (UC = unit cell)
    \item
      all $\epsilon^{\rm SC}({\bf k}_i)$ are (almost) equal to some
      $\epsilon^{\rm UC}({\bf k}_j)$ if ${\bf k}_j$ refolds into ${\bf k}_i$
    \end{itemize}   
  \end{itemize}
\end{itemize}

{\bf Step-2:}
\vspace{-1em}
\begin{itemize}
\item File \file{pw.graphene3x3-O.relax.in} is a modified version of\\
  \file{pw.graphene3x3.scf.in} with:
  \begin{itemize}  
  \item \var{calculation='relax'} for structural optimization and a new
    namelist \card{\&IONS}
  \item \var{ntyp=2} (2 types of atoms), \var{nat=19}
    (19 atoms in the cell)
  \item \card{ATOMIC\_SPECIES} card with 2 species of atoms and
    pseudopotentials
  \item \card{ATOMIC\_POSITION} card with 19 initial positions
    (C--O distance $\sim 1.5$~\AA)
  \end{itemize}
\item Run the structural optimization and analyze the output
\end{itemize}

%%%%%%%%%%%%%%%%%%%%%%%%%%%%%%%%%%%%%%%%%%%%%%%%%%%%%%%%%%%%% 
\head{2. How to perform variable-cell relaxation: hcp-Zinc}
\rightfooter{Example: \file{Day-2/example2.vc-relax}}
%
Zinc displays a hcp (hexagonal-closed-packed) crystal structure, hence
it has two lattice parameters $a$ and $c$. The unit-cell lattice
vectors are:\\
%
\centerline{$\displaystyle
  {\bf a}_1=(a,0,0), \quad {\bf a_2}=(-{a\over 2},{a\sqrt{3}\over 2},0),
  \quad {\bf a}_3=(0,0,c)$}
%
This lattice can be described as:\\
$\bullet$ \var{ibrav=4}, \var{A=$a$}, \var{C=$c$}, {\em both in \AA,
  not a.u.}, as in file \file{pw.Zn.scf.in}\\
$\bullet$ or \var{ibrav=4}, \var{celldm(1)=$a$}, \var{celldm(3)=$c/a$},
as in the file \file{pw.Zn.vc-relax.in}

For hexagonal lattices one needs to optimize two lattice
parameters. This can be done either {\em manually} or by using the
{\em variable-cell relaxation}. (See also \file{README.md}).
\vspace{-1em}
\begin{enumerate}
\item {\bf Manual way:} to optimize the $a$ and $c$ lattice
  parameters, one need to perform a 2D scan over the two
  parameters. With PWTK this can be achieved with the following
  snippet (full script is available in
  \file{Zn-scan.pwtk}):
  \vspace{-0.5em}
  {\codecolor
\begin{verbatim}
foreach A [seq 2.4 0.1 2.8] {
   foreach C [seq 4.8 0.2 5.6] {
      SYSTEM " A = $A , C = $C"        
      runPW pw.Zn.scf.$A.$C.in
   }
}
\end{verbatim}
  }
\item {\bf Variable-cell relaxation:} This is a more convenient
  option. An example of how to perform variable-cell relaxation is
  provided by the input file \file{pw.Zn.vc-relax.in}. Notice:
  \begin{itemize}
  \item \var{calculation = 'vc-relax'}
  \item \nml{\&IONS} and \nml{\&CELL} namelists after the \nml{\&ELECTRONS}
  \end{itemize}
\end{enumerate}
To run the calculation, execute:\\[0.5em]
\exec{pw.x -in pw.Zn.vc-relax.in > pw.Zn.vc-relax.out}

Inspect the output (file: \file{pw.Zn.vc-relax.out}) and notice that:
\vspace{-1em}
\begin{itemize}
  \item several scf steps are performed, forces (zero by symmetry) 
    and stresses computed
    \vspace{-1em}
  \item the energy and the stress decrease as the minimum is approached
    \vspace{-1em}
  \item a final scf step is performed with plane waves computed for
    the final cell
    \vspace{-1em}
  \item the final cell is printed after the last \texttt{CELL\_PARAMETERS}
    card
  \end{itemize}
  Compare optimized parameters estimated from the {\em manual} 2D-scan
  to those obtained from the  {\em variable-cell relaxation}.

%%%%%%%%%%%%%%%%%%%%%%%%%%%%%%%%%%%%%%%%%%%%%%%%%%%%%%%%%%%%% 
\head{2. Variable-cell relaxation (II): molecular crystal of Urea}

While in the preceding example, the forces were zero by symmetry, in
this example (\file{pw.urea.vc-relax.in}) both unit-cell and atomic
positions are optimized by utilizing the {\em variable-cell
  relaxation} (\var{calculation = 'vc-relax'}).  Beware that it is
computationally heavier than the \file{pw.Zn.vc-relax.in} example.

\begin{center}
  \includegraphics[width=0.4\textwidth]{figs/urea.png}
\end{center}

%%%%%%%%%%%%%%%%%%%%%%%%%%%%%%%%%%%%%%%%%%%%%%%%%%%%%%%%%%%%% 
\head{3. NEB method: saddle points of elementary chemical reactions}
\rightfooter{Example: \file{Day-2/example3.neb}}
%

%%%%%%%%%%%%%%%%%%%%%%%%%%%%%%%%%%%%%%%%%%%%%%%%%%%%%%%%%%%%% 
\head{4. Advanced functionals}
\rightfooter{Example: \file{Day-2/example4.functionals}}

Slides for this exercise are provided by file \file{\href{handson-day2-functionals.pdf}{handson-day2-functionals.pdf}}.
%

%%%%%%%%%%%%%%%%%%%%%%%%%%%%%%%%%%%%%%%%%%%%%%%%%%%%%%%%%%%%% 
\head{5. PWTK: a short tutorial}
\rightheader{\includegraphics[width=2.5cm]{figs/pwtk.png}}
\rightfooter{Example: \file{Day-2/example5.pwtk}}

PWTK is a Tcl-based scripting interface for Quantum ESPRESSO. It aims
at providing a flexible and productive framework.

\begin{itemize}
\item PWTK web-site is:\\
  \file{http://pwtk.quantum-espresso.org/} ~or~ \file{http://pwtk.ijs.si/}
\item PWTK documentation is available at:\\
  \file{http://pwtk.ijs.si/toc\_index.html}\\
  (see also \file{\href{http://pwtk.ijs.si/pwtk-slides.pdf}{http://pwtk.ijs.si/pwtk-slides.pdf}})
\end{itemize}

\begin{itemize}
\item  Move to \file{Day-2/example5.pwtk} directory. Therein are three
  examples:
  \begin{itemize}
  \item \file{ex1.eos/} -- how to use EOS (equation-of-state) utility
    of PWTK
    \vspace{0.5em}
  \item \file{ex2.O@Al111/} -- how to run many calculations with a
    simple PWTK script
    \vspace{0.5em}
  \item \file{ex3.CO@Rh100/} -- a more exlaborate PWTK example that
    shows how to glue together various calculations. In particular, it
    analyzes the bonding of CO molecule on Rh(100).
  \end{itemize}
\end{itemize}

%%%%%%%%%%%%%%%%%%%%%%%%%%%%%%%%%%%%%%%%%%%%%%%%%%%%%%%%%%%%% 
\head{5. PWTK: hierarchical configuration}
\rightheader{}

\begin{itemize}
\item Main user configuration file: \file{\$HOME/.pwtk/pwtk.tcl}
  \begin{itemize}
  \item executables and how to run them (\cmd{bin\_dir}, \cmd{prefix},
    \cmd{postfix})
  \item special directories (\cmd{pseudo\_dir}, \cmd{outdir}, \cmd{wfcdir})
  \end{itemize}
\item Project-based configuration (via \cmd{import}), e.g,:  
  \begin{itemize}
  \item \file{Day-2/example5.pwtk/common.pwtk}\\[0.3em]
    Specification of default input data that are common to all examples in
    \file{example5.pwtk/}, e.g., cutoff energies, list of
    pseudopotentials, etc.
    \vspace{0.5em}
    {\small
      \begin{itemize}
      \item {\bf Beware:} for each specific calculation PWTK filters atomic
        species and uses only those that are actually used. Due to that
        the index of species can change from case to case, hence for
        {\em ntyp}-type variables do not use numeric indices, but use
        atomic labels instead, e.g.:\\
        \var{starting\_magnetization({\bf 1}) = 1.0} ~~~({\red\em not recommended})\\
        \var{starting\_magnetization({\bf Fe}) = 1.0} ~~({\green\em recommended})
      \end{itemize}
    }
    \vspace{0.5em}
  \item in a given example the project-based \file{common.pwtk} is
    then imported via, e.g.,
    \file{import ../common.pwtk}
  \end{itemize}
\end{itemize}

%%%%%%%%%%%%%%%%%%%%%%%%%%%%%%%%%%%%%%%%%%%%%%%%%%%%%%%%%%%%% 
\head{5.1 PWTK: EOS utility}

\end{document}
